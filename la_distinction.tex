\section{La distinction -- Pierre Bourdieu}

La thèse principale de cet ouvrage, est que l'intense déterminisme social qui
sous-tend les goûts et les jugements de chacun a pour déterminant la volonté
(ou non) de distinction par rapport à un groupe social.
Voici quelques éléments intéressants expliqués dans cette théorie. Le vocabulaire
est celui du bouquin, ça fait un peu marxiste mais il faut bien mettre des mots
sur les choses.

\begin{itemize}
  \item Empiriquement, les membres de la petite et grande bourgeoisie qui dans
  leur jeunesse affichent une volonté de distinction par anticonformisme
  finissent généralement par se ranger à un jugement plus conservateur avec
  l'âge. Cependant, Bourdieu soutient que cette droitisation n'est pas liée à
  une quelconque sagesse ou expérience acquise avec les années, mais plutôt à
  la superposition de deux effets. Les individus ayant réussi dans leur vie et
  ayant rejoint la fraction dominante de la classe dominante avec l'âge utilisent
  les positions conservatrices justement pour protéger leurs intérêts. Au
  contraire, les individus ayant suivi des trajectoires sociales que l'on
  pourrait qualifier d'échec au regard de leurs aspirations initiales utilisent
  leur droitisation pour rejetter l'anticonformisme qui les a mené dans cette
  position.
  \item Bourdieu présente un diagramme présentant le taux de fécondité en
  fonction de l'origine sociale. On remarque que la fécondité est élevée chez
  les classes populaires et la bourgeoisie mais par chez la petite bourgeoisie.
  Ceci montre la stratégie implicite de la petite bourgeoisie de concentrer son
  capital économique au travers de générations (enfants unique) afin de maximiser
  les chances d'ascension sociale. Cela a beaucoup raisonné avec mon histoire
  familiale (du côté de ma mère, que des enfants uniques, grande famille du côté
  de mon père, dont la mère est de la grande bourgeoisie brésilienne et son père
  des classes populaires corréziennes).
  \item Fascination de Bourdieu pour la fraction de la petite bourgeoisie qui
  joue le plus le jeu méritocratique affiché par la société, et qui passe sa vie
  à se sacrifier en échange d'une hypothétique ascension sociale. Peu de dépenses,
  peu d'enfants, peu de plaisirs, morale rigoriste ; c'est une fraction qui a
  un peu disparu du champ sociétal depuis les années 1980 pour être remplacée
  par ce que Bourdieu appelle la nouvelle petite bourgeoisie marquée par une
  ouverture d'esprit très limitée sur quelques sujets et une soumission complète
  aux intérêts marchands. Cela m'a fait prendre conscience du contrôle social
  très puissant opéré même dans les sociétés que l'on dit libérales ; cependant
  quelle société veut-on vraiment ? Veut-on une société de petits bourgeois ascétiques
  et rigoristes ou une société de grands bourgeois ?
  \item La classe dominante est elle-même extrêmement fractionnée et il est
  important de reconnaître les fractions dominées et dominantes de la classe
  dominante. Est-il possible de se retrouver dans la fraction dominante sans
  en posséder l'état d'esprit, le jugement qui ne peut s'acquérir que par une
  éducation qui baigne dans ce milieu ? Beaucoup d'exemples de ceci donnés par
  Bourdieu en termes de consommation culturelle.
  \item L'autodidacte est toujours jugé sur ce qu'il ne connaît pas, il n'a pas
  le droit d'ignorer comme quelqu'un qui serait passé par une éducation formelle
  où les savoirs sont hiérarchisés. On a beau y mettre autant d'efforts que
  l'on veut, être à l'aise dans un milieu ne peut pas se faire à coup de
  connaissances pures. C'est le concept de bonne volonté culturelle, qui est utilisé
  par les petits-bourgeois pour se distinguer de leurs semblables mais cela ne
  trompe par les bourgeois.
  \item Mon parcours personnel reflète un embourgeoisement et le passage par un
  milieu très bourgeois (l'X) m'a poussé moi même vers des habitudes culturelles
  et de vie que je n'avais pas du tout avant. La théorie du bouquin explique
  donc parfaitement beaucoup de choses qui résonnent avec mon parcours personnel.
\end{itemize}
