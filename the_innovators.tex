%!TEX root = ./reading_notes.tex

\section{The Innovators -- Walter Isaacson}

This book follows the great adventure of computing, networking, the Web and social media
-- Sidenote: funny how social networks became social media now --
from the 1850s to 2014.
Interestingly, there's little mention of Facebook and Twitter: the author didn't foresee opinion based truth.

Things to remember, and ideas they spark:

\begin{itemize}
    \item Innovation comes from good teams. 
    In a good team, often you need a visionary and a good product manager. 
    Note for me (GeoffNN): I'm not a good product manager.
    Definitely a networker/team builder/connector.
    And rather good at starting projects and having ideas.
    \item Building stuff is important.
    Shipping it to the right crowd is even more important: they can help build/curate stuff and feedback is gold.
    Google is collaborative: the only input of the search engine is how to weight contributions (links) $\implies$ PageRank.
    Seeing Google this way changes everything.
    Organizing other people's contributions in a meaningful and easy to understand way is more than gold.
    \item Hiring is important.
    Your company/unit's culture is the one of the people in it.
    You can't change it other than by changing the people (hiring and firing).
    \item Be ambitious.
    And focused.
    Be monomaniac for a few weeks to actually build a version that sort of works.
    Than build on that.
    Hire, get a community to help.
    Whatever.
    \item If going commercial, keep license of what you make.
    Don't sell exclusivity.
    \item People are important.
    Products are for people.
    Everything we make should help people, and people are never the issue: the user crowd is always right.
    Systems must be easy to use.
    Iterate.
    Get feedback.
    Stay open minded.
    \item Teams are important.
    Yours, but not only: help others you believe in build good teams.
    You can't hire everyone good that you know.
    Internship programs are amazing for this: they create networks of good people, and cross pollinates different institutions.
    \item Investing is important.
    In particular time: always collaborate.
    Hang out with smart people.
    You'll get ideas and motivation.
    \item Keep humanities in mind. We're augmenting intelligence.
    This should be for a good purpose.
    Caveat about the Defense industry: protecting soldiers giving their lives, and protecting people in general is a good goal (GeoffNN)
    \item Status now: good network of good people.
    Not enough building things that work because of lack of motivation and \textbf{focus}.
    Why? Usefullness.
    I'm a practical oriented person (despite what people say) well versed in theory. 
    \item I like where I'm going these days.
    Towards more collaboration -- always good in the long run.
    If people steal my ideas, it's only that I didn't convince them to be on my team, or that I wasn't good enough delivering.
    Fighting big guys on their turf (large computation needs for ex in research) is stupid.
    Having better ideas, concepts, MVP and TEAM to make them is where a small team's value is.
    Being acquired is fine.
    \item I want to do this.
    I actually hate having a boss -- but I'm really open to learning.
    \item On this note: take a boring job at some point.
    I want to feel frustrated and the need to build something of my own.
    Challenging smart  people and hogging their time is how big companies dominate the game.
    Even more so than paying them; although for the more risk averse that's an easy way too.
    \item All these incredible systems are amazingly young.
    And started with nothing.
    And developped randomly: a VC suggested advertising to Page and Brin at breakfast.
    This changed the Internet and a lot of people's lives, without them knowing it.
    \item Build a content/user centered social media?
    Medium is one.
    Lacks hyperlinks/easy feedback.
    Author (Isaacscon) suggests a collaborative e-book.
    How can this make money? How could this be done? 
    \item The interaction between humanities and engineering is key.
    We build roads to connect people and sell stuff.
    We built the Internet to share ideas -- it stemmed from Defense and Research.
    Al Gore opened it to everyone and commercial ventures.
    People want to sell stuff, make money, and bring their ideas and content to other people.
    People get a wiki-high when they post updates that everyone will read.
    ASMR and so many other YouTube subcultures yield human gratification
    -- and money, as a by-product.
    Content has worth: if you make something people like, you should make money somehow.
    \item Government, industry, academia, and people, i.e. \textbf{everyone} (i.e. the crowd) working together is the best thing that can happen.
    Their roles are different, different kinds of people work in each setting, but if everyone has the right spot and works with the right people, amazing things can be built.

\end{itemize}