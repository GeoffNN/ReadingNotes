%!TEX root = ./reading_notes.tex
\section{Introduction}

This, for now, are my personal reading notes.
I just finished the Innovators by Walter Isaacson, which has elicited a lot of excitement on my part.
A few of the reasons for this are the following: it is a tale of wonderful systems built in a relatively short time.
It is also a reminder that people are capable of great ideas and of building great things when they are in the right environment, and when they work with the right people for them.
It's also a testimony to the power of government, industry, academia and \textbf{everyone else} both working together, and competing.

This reading, and the fact that I have a bunch of new books on my reading list thanks to Alice, has spurred me into starting this reading notes series.
The idea is to keep short, bullet-point teachings from books I read, and posting them on Github.
I hope this will allow for comments, add ons, and discussion of these points by other people. In typical blog fashion, I'll be writing about teachings for me personally, but in wiki fashion, I hope that I won't be the only one editing these documents. 
ull requests will allow to do this easily, and git will allow both moderating and history archiving.
You can also just fork this repo to keep your own reading notes privately.

If you are reading this, please take a look at the repo.
If you have read a book that's in here, feel free to add your personal comments.
If you've recently read an exciting book, feel free to add a .tex file for that book with your comments.

Hope it helps and looking forward to your input.

PS: I really should be writing this in Markdown, but I don't know yet how to include different files so that README will include everything.
I'll figure it out soon, promise.

Two ways to read and edit this document:

\begin{enumerate}
    \item If you know LaTeX, you can just read the source files.
    It's mostly text anyway.
    \item If you don't, I'll post a current version of the PDF on my personal webpage.
    On the other hand, edits will have to be made using LaTeX, so I guess you'll either have to learn the basics, or just edit inline using Github and the examples given by current reading notes.
    Have fun. :)
\end{enumerate}

\subsection{Contribution rules}
\begin{itemize}
    \item Add \textit{\%!TEX root = ./reading\_notes.tex} at the beginning of every new book file.
    \item \textbf{One sentence per line. This allows for easy merging of edits. Thanks!} 
\end{itemize}

Pull requests that don't respect these rules will be rejected. If you have any doubts, just check current files.