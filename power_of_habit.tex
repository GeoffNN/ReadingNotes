%!TEX root=./reading_notes.tex
\section{The power of Habit -- Charles Duhigg}

This book is concerned with how your habits can be used to get great results in what you want, and how to change your habits.
This can be seen at both a personal and institutional level, where organization habits are a mix between official routines and organization culture.

The goal is to make it unnessary to think to do great things: these things should be automatic.

\begin{itemize}
    \item Having studied in several difficult programs, and practicing martial arts, I understand that habit building is basically drilling. 
    \item Building a habit and not having it break down under pressure takes 4 things: a cue, an action, a reward; and a strong belief in the rightfulness of the habit.
    \item Not all habits are equal: keystone habits are habits that change your way of life/your structural organization.
    Exercising regularly, or being having one important goal in mind will yield subhabit creations.
    Having to study for long hours changes your sleep/wake up routing for example.
    Prépa: having to be able to concentrate for long hours meant everything else had to become automatic.
    Waking up, showering, having breakfast, sleeping at a certain time.
    It all came from weekly homeworks having to be handed in without thinking.
    From focus emmerges structure.
    \item Some habits trigger belief; e.g. being focused on worker security for Alcoa (Aluminum company of America).
    Making it easy for good security protocols to emerge and fixing everything fast made the whole company much more nimble and aligned management and workers' interests.
    \item For me (GeoffNN), showering and then having breakfast immediately are keystone habits. 
\end{itemize}